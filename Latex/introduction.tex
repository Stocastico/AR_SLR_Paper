\section{Introduction} \label{sec:introduction}
Digital transformation is profoundly impacting and disrupting every facet of society, and education is no exception. In the last two decades, several technologies and techniques have impacted the education industry, being the most notorious the irruption of \gls{MOOC}, the widespread adoption of \gls{IWB} and, more in general, the success of \gls{TEL}. 

Despite being first introduced in 1992 as a training tool for Air Force pilots \citep{caudell1992augmented}, it took many years before \gls{AR} was applied in schools as a tool to facilitate learning. Nowadays, thanks to the widespread adoption of devices that support \gls{AR} applications, as well as the availability of software libraries like ARKit\footnote{developers.google.com/ar/}
or ARCore\footnote{developer.apple.com/documentation/arkit/} which greatly simplify and speed-up the development process, \gls{AR} has become a technology which is more and more used in educational settings.

% Add references to previous SLR with details on what they studied
Given its surge in popularity, \gls{AR} has become an active research topic and several systematic studies have been performed to analyse how this technology has been used in educational contexts. Some studies presented an analysis of the advantages and drawbacks of \gls{AR} in generic educational settings \citep{akccayir2017advantages, radu2014augmented, diegmann2015benefits} or provided insights on the status of the technology as well as suggestions for future research \citep{cheng2013affordances, arici2019research, bacca2014augmented, pellas2019augmenting}. Other reviews focused on specific subjects, like \gls{STEM} \citep{ibanez2018augmented, nielsen2016augmented}; on specific topics like \gls{AR}-based serious games \citep{li2017augmented, bartolome2011can, laine2018mobile} or the evaluation of the usage of \gls{AR} in schools \citep{da2019perspectives, chen2017review}. \citet{garzon2019systematic} performed a meta-review of 61 studies to measure the impact of \gls{AR} applications in learning effectiveness, showing that at the moment they only have a medium effect on it. Table~\ref{tab:slrsummary} summarises the content of some of the most comprehensive \glspl{SLR} about \gls{AR} in educational settings.

Since the publication of the seminal paper on collaborative \gls{AR} by \citet{billinghurst2002collaborative}, which first discussed how AR could be used to enhance online and offline collaboration, much progress has been done on providing collaborative tools to \gls{AR} applications. To the best of our knowledge, though, there has been no \gls{SLR} about the usage of \gls{AR} applications as tools to improve collaboration between students as well as between students and teachers, or the possibilities provided by multi-user interfaces as an enabler for improved and accelerated learning. 

Cooperative learning, defined as the instructional use of small groups to promote students working together to maximise their own and each other's learning \citep{johnson1991cooperation}, has long been used as an educational approach to improve the students' learning and performance \citep{johnson2008active, kuh2011piecing}. At the same time, how students engage with technology is often driven by how much they can interact with the different tools. Unfortunately, AR per se is neither an interactive media nor inherently a multi-user tool: it is up to researchers and developers to provide such functionalities in an AR-based educational application. With this work, we aim to evaluate which publications described AR applications that provided interactivity and multi-user functionalities and how they affected the students'\ engagement and their academic performance.

The main contribution of this paper is to provide a \gls{SLR} of the \gls{AR} applications deployed in primary and secondary schools, with a particular focus on the collaborative, multi-user and interactive characteristics of such applications. We decided to consider only the articles published after 2015, since in that year the number of publications related to the application of \gls{AR} in education has seen a huge increase (as it is shown in Fig.~\ref{fig:pappublbg}).

The \glspl{RQ} that we addressed with this study are:
\begin{itemize}
    \item \gls{RQ}1: What collaborative, multi-user, interactive \gls{AR} applications have been used in an educational environment in primary or secondary schools?
    \item \gls{RQ}2: What was the motivation for using \gls{AR} as an educational tool?
    \item \gls{RQ}3: How effective are \gls{AR} applications at improving the students knowledge of a subect? How is this evaluated?
\end{itemize}

\begin{table*}[htbp]
\centering
\caption {Summary of \glspl{SLR} about usage of AR in education.}\label{tab:slrsummary}
\begin{tabular}{|M{4.0cm}|M{0.8cm}|M{3.7cm}|M{1.5cm}|M{4.8cm}|}
    \hline
         \textbf{Study} & \textbf{Year} & \textbf{Purpose} & \textbf{Studies reviewed} & \textbf{Findings} \\
    \hline
    \hline
         Systematic review and meta analysis of augmented reality in educational settings \citep{garzon2019systematic} & 2019 & Identify the status and tendencies in the usage of \gls{AR} in education & 61 & \gls{AR} has a medium effect on learning effectiveness; lack of studies considering accessibility features in \gls{AR} apps \\
    \hline Augmenting the learning experience in primary and secondary school education: a systematic review of recent trends in augmented reality game‑based learning \citep{pellas2019augmenting} & 2019 & Explore the combination of \gls{ARGBL} & 21 & Motivation and enrichment are pillars of \gls{ARGBL}; \gls{ARGBL} compares favourably to traditional learning \\
    \hline
         Augmented reality for STEM learning: A systematic review \citep{ibanez2018augmented} & 2018 & Perform qualitative analysis of the characteristics of \gls{AR} apps for \gls{STEM} learning & 28 & Most apps offer exploration or simulation activities, but usually without providing assistance in carrying out learning activities; similar design features across all studies \\
    \hline
        Advantages and challenges associated with augmented reality for education: A systematic review of the literature \citep{akccayir2017advantages} & 2017 & Identify advantages of \gls{AR} in education and identify current gaps in \gls{AR} research & 68 & Conflicting results regarding cognitive overload of \gls{AR}; low usability is the main challenge of \gls{AR} apps for education \\
    \hline

\end{tabular}
\end{table*}

Besides answering these research questions, we will also discuss the different technologies used by such applications, for example, the hardware required (\gls{HMD}, tablet or smartphone), the way the system tracks information from the real world (marker-based, markerless, location-based), whether the application augments other senses beyond vision, and which design strategies (if any) have been used to make the applications accessible.

\begin{figure}[ht]
	\begin{center}
	\includesvg[width=0.48\textwidth]{figures/papers_over_years}
	\caption{Numbers of papers published per year with topic "augmented reality" and "education" as of 1 October 2020.}
	\label{fig:pappublbg}
    \end{center}
\end{figure}

The rest of the paper is structured as follows. Section \ref{sec:methods} describes the methodological design of the study, including an explanation of the work done to plan, conduct and report the review. Section \ref{sec:results} presents the findings of the systematic review and the answers to the research questions. Section \ref{sec:discussion} discusses the results obtained and suggests possible research lines as well as trends for the future of AR in education. Finally, Section \ref{sec:conclusion} summarises the conclusions of the paper.
% You must have at least 2 lines in the paragraph with the drop letter
% (should never be an issue)