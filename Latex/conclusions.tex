\section{Conclusion} \label{sec:conclusion}
In this paper we presented a systematic review of the literature relative to applications of immersive, collaborative and multi-user \gls{AR} in education. We analysed \papersSelected studies and evaluated their technical characteristics and their advantages compared to traditional teaching tools as well as the impact they had on knowledge retention. We believe that the findings described in sections \ref{sec:results} and \ref{sec:discussion} can be useful for researchers in driving the design of the next generation of AR applications.

With the first \glsfirst{RQ1} we wanted to identify which studies described interactive, multi-user and collaborative AR experiences, and we compared the main features of the AR applications described. Every paper presented AR-based interactions, but only a few applications provided multi-user and collaborative capabilities. Our analysis showed that Unity and Vuforia, the de-facto standard tools for creation of \gls{AR} applications, do not provide researchers and developers the tools to easily include collaboration mechanisms to \gls{AR} applications.

The second \glsfirst{RQ2} aimed to understand the motivation behind the usage of \gls{AR} as an educational tool. In this case we analysed both the motivations presented by the researchers and the results of surveys conducted on students. Even though only few papers provided information in this sense, it appears that the main motivation for using \gls{AR} in schools is to facilitate understanding of abstract concepts and to increase students' engagement.

Finally, the objective of the third \glsfirst{RQ3} was to measure, as objectively as possible, the impact of using \gls{AR} in the classroom. The studies analysed used pre/post tests or comparison with control groups to assess the usefulness of \gls{AR} and, in general, they showed that making use of \gls{AR} applications provides a small but statistically significant improvement to compared to the scores obtained by the test group. Only the work of \citet{lin2016effect} presents an analysis of the retention of the topics learned through \gls{AR} on a time span of more than two months. As most of the students who participated in the tests did not previously use \gls{AR} applications, there is a concrete risk that the novelty effect introduced a recency bias, by increasing user engagement and knowledge acquisition, indirectly leading to better test scores.  

\subsection{Limitations of this study}
This review was limited in that it examined articles from four databases: IEEExplore, ISI Web of Science, Scopus, and Springer, from 2015 to 2020. The articles in these databases are considered to have a high impact on the field; however, the latest technical reports and business demonstrations of AR in education were excluded from this review, which may limit the representation of the state of the art. Although the number of papers included in this review was limited, the selection process was completed using a systematic process in order to avoid bias.