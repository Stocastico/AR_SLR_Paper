\begin{abstract}
Augmented Reality is a technology that enhances human perception with additional, artificially generated sensory inputs to create a new experience which enriches human vision by combining natural with digital elements. Augmented Reality development dates back to the early nineties but it is only in the last decade, thanks to improvements to hardware and software, when it has begun to be rapidly incorporated in several fields, including education. This study presents a systematic review of the literature on the use of augmented reality applications in primary and secondary schools, with a specific focus on collaborative, multi-user and interactive applications. The aim of the study is to investigate the characteristics of such applications, the processes that led to their adoption, and their effectiveness in enhancing the learning experience. This study synthesises a set of \papersSelected publications from 2015 to 2020 and performs a qualitative analysis of their content. The review describes the current state of the art in research in Augmented Reality for education and provides future research lines, as well as trends for the future of such applications in educational settings, analysing the relevance of the multi-user interaction challenge within the Augmented Reality ecosystem.
\end{abstract}