\subsection{Overview of reviewed studies}

Of \papersSelected studies reviewed, about 70\% of them (\papersAfterTwentyEighteen articles) were published in 2018 or afterwards. The vast majority (65\%) of the AR apps analysed cover STEM subjects, while about 20\% of the studies cover Humanities and Foreign language subjects. The remaining articles cover specific subtopics such as sustainability, creativity and social interactions or do not specify the subject. Fig. \ref{fig:subjects} summarises the subjects covered by the AR apps analysed in this \gls{SLR}.

% \begin{figure}[ht]	
% 	\begin{center}
% 	\includesvg[width=0.9\textwidth]{figures/subjects}
% 	\caption{Subjects covered in the studies analysed.}
% 	\label{fig:subjects}
%     \end{center}
% \end{figure}

Regarding which AR type is used in the classroom, marker-based solutions (either image or QR-code based) are the most used, as almost two thirds of the studies described apps using markers as the exclusive source of the augmentations. 
Some studies describe applications using multiple types of \gls{AR}, usually a combination of markers and object detection based methods. Other types of \gls{AR} such as markerless or location based are seldom implemented, as they were used only in 9 and 2 articles, respectively. Fig. \ref{fig:artech} summarises the types of AR used by the articles analysed in this \gls{SLR}.

% \begin{figure}[ht]	
% 	\begin{center}
% 	\includesvg[width=0.9\textwidth]{figures/AR_technology}
% 	\caption{Different types of AR used in the studies analysed.}
% 	\label{fig:artech}
%     \end{center}
% \end{figure}

With reference to the hardware required to experience the AR apps and the software used to develop them we notice a similar pattern. Most of the studies describe apps which have been developed for smartphones or tablets using the Unity\footnote{unity.com/} framework, often in conjunction with the Vuforia\footnote{developer.vuforia.com/} \gls{SDK}. Some studies, usually the oldest ones, describe systems using projectors or PCs with depth sensor cameras such as Microsoft Kinect\footnote{developer.microsoft.com/en-us/windows/kinect/}. Only \hardwareHMD articles describe apps which require \glspl{HMD} or smart glasses \citep{wei2018improving, oh2016designing, oh2017hybrid, kum2019ar, khan2018mathland, matsutomo2017computer}. This might be due to the higher cost of such devices and their consequent limited adoption compared to smartphones or tablets.

Using web technologies for the creation of AR application is still the exception rather than the norm: despite the availability of a javascript library such as Three.js\footnote{threejs.org} and frameworks such as A-Frame\footnote{a-frame.io}, only the works of \citet{abriata2020building} and \citet{protopsaltis2016quiz} provide augmented content that can be consumed through the browser.
Somewhat surprisingly, very few studies rely on the libraries produced by Google and Apple (ARCore and ARKit), which were developed to provide advanced \gls{AR} functionalities for smartphone and tablets. Usage of specialised \gls{CV} libraries or \gls{DL} framework is also very low, which probably means that researchers prefer to use the functionalities provided by Unity. Statistics about software usage may be skewed, though, as about one third of the studies did not provide information about it.

Fig. \ref{fig:hardware} and \ref{fig:software} summarise the hardware required and the software used by the apps analysed in this \gls{SLR}. The total in this case does not sum up to \papersSelected since the same application could support more than one device and likewise it may have been developed using several software libraries.

% \begin{figure}[ht]	
% 	\begin{center}
% 	\includesvg[width=0.9\textwidth]{figures/Hardware_supported}
% 	\caption{Device types supported by the AR applications.}
% 	\label{fig:hardware}
%     \end{center}
% \end{figure}

Unfortunately, researchers very rarely publish their code alongside their peer reviewed publication. Of all the studies we analysed, only four \citep{mylonas2019educational, laviole2018nectar, ManriqueJuan2017APA, abriata2020building} publicly released the source code of their application. In some cases the researchers published the application for free on Google Play or the App Store. Although in principle this allows other researchers to test the application, without releasing the source code this is impractical, as it is very rare that the application can be used without some form of adaptation (for example, translation of the content, inclusion of new multimedia elements or adjustments to the school curricula).

% \begin{figure}[ht]	
% 	\begin{center}
% 	\includesvg[width=0.9\textwidth]{figures/Software_used}
% 	\caption{Software used to develop AR applications.}
% 	\label{fig:software}
%     \end{center}
% \end{figure}