\section{Discussion} \label{sec:discussion}

This study shows that the research community is very active in investigating how \gls{AR} applications can improve education and facilitate students' understanding of difficult concepts.
Even though collaboration and participation by students is often seen as a key towards improving knowledge retention, we still see a lack of support for cooperation mechanisms in AR applications for education: of the \papersSelected studies analysed, only \papersMultiuser described multi-user application and only \papersCollab employ some sort of collaboration between users. \gls{ARGBL} is also quite uncommon, as only \papersGames articles describe applications which implement gamification concepts. 

By reviewing the existing literature we have also identified several issues that are preventing the widespread adoption of collaborative \gls{AR} in the classroom:
\begin{itemize}
    \item Lack of authoring tools: with the exception of the works of \cite{lytridis2018artutor} and \cite{whitlock2020mrcat}, the applications described do not make use of an authoring tool that simplify the creation of an AR experience. This means that every AR application has to be developed from scratch, requiring longer development times and multiplying the amount of work required from the developers.
    \item Lack of standardisation for the description of AR experiences: of all the papers we analysed, none of them mentioned using a standard for the description of how AR is used in the application. This is mainly due to a lack of specific standards, as the IEEE ARLEM standard \cite{arlem2020} for AR-based learning experiences was only released in February 2020, while the ETSI Augmented Reality Framework\footnote{www.etsi.org/committee/1420-arf} for the interoperability of AR components has not been published yet. We believe that adoption of these standards will drive and simplify the development of \gls{AR} applications for education, as well as foster interoperability.
    \item Availability of 3D content for education: a few repositories where users can freely download 3D objects already exist, but there is a lack of 3D content specialised for education purposes. Although there are currently efforts being made to solve this issue \cite{masneri2020work}, it does severely hinders the possibility of quickly creating new AR apps for primary and secondary schools.
    %: very often the apps never leave the prototype stage and they are not turned into a fully-fledged product or used further in the classroom.
    \item Code publication: another issue with most of the studies we reviewed is that only a small fraction of the authors published the code of the \gls{AR} application. This means that other researchers cannot build upon the results of previous researchers: even for the more interesting and highly cited articles there will be no follow up work, with the exception of that from the original authors.
\end{itemize}

We noticed that studies claiming to have a stronger positive impact on educational achievements are the ones where the \gls{AR} application is part of bigger learning environments. We believe that providing automatic logging functionalities, for example through xAPI \cite{kevan2016experience}, a teacher dashboard where the educator can track the progress or the grades of each student and a set of tools for improving communication capabilities could go a long way to better integrate \gls{AR} applications in standard schools curricula. Using xAPI could simplify the application of learning analytics techniques for the analysis and improvement of students' learning. This is especially the case for distance learning, in which the students are not in the same physical space as the teacher or other students but are following their classes remotely.

On the technical side, researchers are slowly adopting the latest advancement in technology, but the majority of the studies analysed are still focusing on more limited \gls{AR} functionalities, for example marker-based systems. The implementation of \gls{AR} applications that make use of \gls{EAI} or which are based on web technologies such as WebXR\footnote{www.w3.org/TR/webxr/} is currently limited because only the most recent devices have hardware capable of supporting them. Nonetheless, we believe these are key technologies that enable more immersive experiences and facilitate collaboration. 

Most of the studies we reviewed, with the exception of the works described in \cite{chen2018application, kenoui2020teach, mikulowski2020multi}, focus on vision-based augmentations. Although it is clear that students rely predominantly on sight to collect and process information, providing other types of augmentations such as haptic or audio is worth investigating, since these could make the user experience more immersive and they could improve accessibility of \gls{AR} applications for students with sight impairment.

None of the studies explored the possibility of using multi-user AR application for distance learning. The apps described by \cite{oh2017hybrid} and \cite{lopez2020emofindar} use PUN\footnote{www.photonengine.com/pun}, a network library that enables communication across different devices, but the applications require that the users share the same physical space. Especially after the prolonged lockdown due to the Covid-19 pandemic, newer technologies should provide AR apps with capabilities for the students to share the same experience even though they are not in the same room. This would be useful for teachers, who could make remote lessons more engaging, and for students, who would have the chance to work together with other schoolmates even when they are at home.  

Regarding the effectiveness of AR applications in the classroom, the majority of the studies present an evaluation of the \gls{AR} solution described. There are great differences between the questions for teachers and students in the user surveys, but in general users find \gls{AR} a successful educational tool which is both useful and engaging. The most common critiques identified refer to the user friendliness of the application and the errors in identifying the markers. More specifically, the users complained about the difficulty of navigating through the UI, due to its lack of consistency and about the difficulty of identifying and tracking the markers in poor lighting conditions or when the camera was not close enough.