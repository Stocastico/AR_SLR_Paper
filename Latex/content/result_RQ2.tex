\subsection{Motivation for using AR as an educational tool}

This subsection addresses the second research question. While the studies reviewed do not usually motivate the choice of the particular application presented in the articles, they do present however, several advantages provided by AR in the classroom. The main advantage provided by AR is that it can integrate seamlessly with the real world, especially for markerless applications that can interact with objects or printed material already available in the classroom. This encourages student engagement and minimises the time required to learn how to use the technology, allowing the students to spend more time learning the subject, as shown by \cite{thamrongrat2019design}. A more recent work by the same authors \cite{233-10.1145/3441000.3441034} shows that using gamification concepts in AR significantly impacts the students results.

Another advantage provided by AR is that this technology does not require the existing curriculum to be remodeled, rather it can be used as a tool to stimulate interest or to supplement existing pedagogical materials by simply adding more contextual experiences. \cite{pombo2018edupark} mention that using an AR app improves the engagement and interest of the students visiting an urban park by providing information that would otherwise be available only on textbooks.

AR is also a powerful tool for visualisation and animation, especially for STEM subjects, as it offers several advantages for displaying \emph{3D} or \emph{3D+t} information (i.e., tridimensional data changing over time) in comparison to books, blackboards or videos. The work of \cite{cao2019hand} describes an application for learning 3D geometry where the user can interact with 3D objects with their hands. The fingers are tracked with a Leap Motion Controller\footnote{www.ultraleap.com/product/leap-motion-controller/} while a set of markers are used to generate the augmented content. In \cite{246-10.1145/3379350.3416155}, the authors use advanced features provided by ARKit (such as joint detection) together with object tracking technologies to provide interactions and visualizations through sketches drawn on the device.

In the context of collaborative and multi-user applications, AR similarly helps to provide new opportunities for students to learn how to communicate and collaborate with one another, as well as to inspire empathy and to teach the importance of teamwork \cite{hill2013classroom}.

Some of the reviewed studies used AR applications as radically new tools that could improve skills and grades of children with mental or developmental disabilities: \cite{luna2018words} describe an application that helps students with \gls{ADHD} improve their English literacy skills. Similarly, the work of \cite{chen2019effects} uses AR together with concepts maps to teach kids with \gls{ASD} different types of social cues designed to help them when meeting people. \cite{takahashi2018empathic} designed a large scale AR and projection system, modifying the gymnasium of the school, to create a learning game for children with \gls{ASD}, which intends to keep their attention focused on the content provided.

In \cite{258-beyoglu2020use}, the authors check the effects of using mixed reality applications and how they impact the students' motivation. They show that while such apps do not significantly impact the motivation to learn, they increase the students' motivation for collaborative working, and the results are more significant for AR than for \gls{VR} apps. More in general, AR also compares favourably with respect to \gls{VR} not only because it allows users to perform tasks faster \cite{7833028}, but also because its requirements (namely a stable internet connection and one or more mobile devices), can be provided at a lower cost and the system does not need as much time to set up. Cost is often seen as one of the most important factors limiting the access of newer technologies, so in this sense AR is often seen as a better tool in comparison with VR or expensive hardware such as laptops and projectors.